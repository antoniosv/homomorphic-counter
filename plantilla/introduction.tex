\chapter{Introduction}
\label{intro}

Communication has always been a basic human need, and the means to do it is especially linked to the available technology. Thanks to the advances in techonology, it has become more common to exchange information with our peers remotely. Whether it is our personal information or current location, there are many scenarios where it is necessary to pass that information to another point. It becomes an issue when the information is considered to be \emph{sensitive}, and thus, the confidentiality of it must be protected, so that no one other than the sender and the intended receiver get to learn about the sensitive information. The medium used by both the sender and receiver to transmit any piece of information is called a \emph{channel}, which is often considered to be unsecure. It is assumed that apart from the sender and receiver, there might be a third party that tries to eavesdrop on the communication lines. In this case, it becomes important to protect the confidentiality of the information shared between the sender and receiver. 

For the scenarios where confidentiality becomes important, \emph{cryptography} becomes an attractive solution, so that encryption techniques are used to put the information in some kind of \emph{safe}, preventing any eavesdropper to learn of its contents. The issue at hand becomes even more interesting when the safe containing the private information does not go directly to the intended party, but is rather stored somewhere in the cloud, ie. the Internet. As long as the proper encryption mechanisms are put in place, the confidentiality is not necessarily compromised if the data were stored in the cloud. However, merely storing static information is not quite interesting: one would want to modify it as the need arises, such as when updating personal data. Such goal calls for \emph{homomorphic encryption}, which is an advanced technique used to modify the already encrypted data without compromising its confidentiality.

There are many instances where it is required to send information to someone else. To sign up for a service, for instance, several pieces of information are required. Usually, name and email address are required as basic pieces of information. Depending on the service, telephone number and personal address might also be required. In this case, the subscriber might actually be worried about how safe his information is being kept. He would feel safer if he had some kind of proof that vouched for the confidentiality of his data. This also affects the service, as less people would sign up for the service if they had no means to protect the data. There are several approaches to protect sensitive data, and one of the most effective ones is based on cryptography.

Cryptography is the study of techniques that enable secret communication, such as ciphers, that is, encryption and decryption algorithms to be used on sensitive data. Commonly, these algorithms are used to protect the data from eavesdroppers that try to pry on it. Ciphers can be seen as the means of putting sensitive data into a box with a lock, and only those with the appropriate secret key are able to access the contents of the box. It becomes troublesome to manage the secret key, because once it has been found by someone, any piece of data encrypted with the same lock becomes vulnerable to unauthorized access. Often, there is also a secure channel that is mostly dedicated to the exchange of secret keys; however, it is often limited and has certain constraints that makes it unfeasible to use it for anything else.

It has become increasingly common to delegate computing tasks to cloud services in order to save resources. For instance, these services might be used when it cannot be afforded to build and maintain a data center for the storage and computing needs that an investor might have. This is especially the case when the level of activity is seasonal: this is, requiring more processing and storage capabilities than usual during certain periods of the year. Therefore, a common solution is to make use of cloud storage and computing services, so that the costs only increase during these usage peaks. This way, the budget is spent as much as the resources are being employed, while saving configuration and maintenance costs. However, as Srinivasan \cite{Srinivasan:2012:SCC:2345396.2345474} points out, allowing the cloud service to make use of the data has raised concerns on security, since it is hard to trust that the cloud provider will not look at the data and do something with it.

A typical solution to this scenario consists in encrypting the data before storing it in the cloud; therefore, assuring its confidentiality. It is a great solution when it is only needed to read the data, without actually making any changes on it. In order to view the encrypted data, it has to be downloaded and decrypted using the proper secret key. However, when it is required to modify the encrypted data in some way, it requires to be downloaded and decrypted back to its plaintext form, before modifying it. Once it has been modified, the data has to be re-encrypted and re-uploaded to the cloud, in a potentially slow manner, depending on the size of the data. Abadi \cite{abadi2009data} explains that this approach turns out to have high overhead costs caused by the transfer of the encrypted data back and forth between the user and the cloud. In other words, it is very bandwidth intensive to encrypt and decrypt complete tables or columns out of the cloud every time some kind of change or analysis must be performed.

In scenarios where the information is not considered to be sensitive, it would be appropriate to make use of cloud computing services directly, as it would cause no harm to grant access to the data that is to be used if it were to be considered public. However, real life applications often imply sensitive data, and it should not be handled trivially, as there is no telling what kind of use an eavesdropper might give to the data.

A simple sounding, yet complicated solution to this problem would be to manipulate the encrypted data in some way such that the contents are not revealed, but on decryption, ends up being correct. In other words, to perform some kind of computation on the encrypted data, even if the private key is not known, so that when it is decrypted, the change is taken into account and the result is the same as if the operation had been performed on plaintext data. For some time, it had been thought that doing something like that was not possible, as ciphers often consisted of permutations and substitutions. However, Rivest et al. \cite{rivest1978data} noted an interesting property found in the public-key RSA algorithm (named after its creators) that proved otherwise. Such property is called ``homomorphic'', and has since then gone through a lot of research, resulting in many schemes such as ElGamal \cite{ElGamal:1985:PKC:19478.19480}, Goldwasser---Micali \cite{Goldwasser:1982:PEA:800070.802212}, Benaloh \cite{benaloh1994dense}, Paillier \cite{Paillier:1999:PCB:1756123.1756146}, among others.

\section{Problem Definition}

Even though there are many homomorphic encryption schemes that have been created to allow for computations on encrypted data, so far, they are not considered fast enough to build efficient secure cloud computing solutions. In recent years, a couple of homomorphic cryptosystems implementations have surfaced, namely: HElib \cite{helib} and FHEW \cite{fhew}. However, they are best thought of low-level building blocks for homomorphic encryption. Even though both implementations are based off optimized versions of the proposed homomorphic schemes by Brakerski et al. \cite{cryptoeprint:2011:277} and Ducas and Micciancio \cite{fhew} respectively, there have not been attempts to use these libraries to build solutions in cloud computing. As such, it has not been shown if with current implementations it is still unfeasible to build a software solution using a homomorphic scheme. In this case, feasibility is put in terms of processing time and required storage size. 

\section{Motivation}

P{\"o}tzelsberger \cite{potzelsberger2013kv} reports that there are many areas in which homomorphic encryption could be used, such as the medical, marketing, and financial fields. Until now, there was no way to make a concrete implementation of a solution related to these areas, since available schemes were either too limited or too slow. Using HELib, a library that provides fully homomorphic encryption, it would be possible to reevaluate the situation and show to what degree it would be feasible to make a solution based on homomorphic encryption.

The proposed solution consists in taking the functionalities provided by HElib, and build a client-server based software that aims to address the main scenarios found in the proposed case study. It is better thought of as proof of concept that shows how homomorphic encryption can be used in an environment that emulates cloud computing. On this attempt, basic communication functionalities are explored, as well as the flow in which the tasks are divided between the client and the server. Particular attention is put on the performance of this application, notably execution time of the homomorphic evaluations, as well as the storage required for certain items.

The idea behind this attempt is that by showing a compelling example of how homomorphic encryption can be applied in a simple scenario, cloud service providers and developers might start considering how to apply homomorphic encryption in other ways and mediums, and thus, expanding on software that makes use of it.

\section{Hypothesis}

Building a client-server based solution using the homomorphic encryption functionalities provided by HElib is feasible in terms of processing time.

\section{Objectives}

\paragraph{General Objective}

Develop a client-server based solution using HElib to perform homomorphic evaluations on encrypted data.

\paragraph{Specific Objectives}
\begin{enumerate}
\item Establish a client-server architecture where homomorphic encryption can be applied.
\item Identify which factors pose a challenge to deem applications of homomorphic encryption as inefficient.
\item Show the use of HElib to setup, encrypt, and decrypt as simple as can be.
\item Collect performance data on the use of homomorphic encryption.
\end{enumerate}

\section{Case study}

Consider the scenario where a household has an expected pattern of activity, ie. empty during the day and non-empty at night, so that the householder seeks to ascertain the number of people inside at any point in time during the day. Assuming the householder has put in place certain sensors around the building so that it detects and counts who comes in and out, he would like to learn the value of the counter remotely. As the householder chooses to store the value in the cloud, he quickly realizes he does not want others to learn of this value, not even the cloud service itself, as to prevent potential burglars to break in when the household is empty. Additionally, he wishes that the counter gets updated at certain intervals, e.g. every 20 minutes. 

A typical approach consists in encrypting the counter value every time it is recalculated, replacing the older value stored in the cloud server. Considering the overhead costs of this process, he considers another potential approach that makes use of homomorphic encryption. The sought solution needs to eliminate the need to reupload the counter value every time it is updated by a change in the number of people in the household. Additionally, the private key should only be in the possession of the householder, so that even the cloud hosting service does not have the ability to learn the value of the counter. Finally, whenever he needs to know of the current value of the counter, he can download the encrypted value and, using the previously generated secret key, decrypt the data to obtain the value of the counter.

\section{Structure}
This thesis is made up of the following chapters: introduction, background, related works, methodology, experiments, experiments and results, and conclusions.

In this chapter, both the problem definition and motivation are posed to build a client-server architecture using HElib to address the scenario of the proposed case study.

In Chapter \ref{background}, necessary basic notation and definition of concepts on cryptography are presented to acquaint the reader with the content presented in the thesis.

Chapter \ref{relatedWorks} presents some works that are related to solutions based on homomorphic encryptography. The characteristics of each work is briefly described and compared with the ones presented in this thesis.

Chapter \ref{methodology} presents the implementation details of the proposed client-server architecture, describing each component and how the functionalities of HElib are put in use. Additionally, the case study is addressed within the scope of the client-server solution as well.

Chapter \ref{results} presents the performed experiments to test certain parts of the client-server implementation related to the use of homomorphic encryption, as well as showing the results pertaining to the processing times and size. Afterwards, it presents conclusions based on these results.

Finally, Chapter \ref{conclusions} presents general conclusions obtained from the results during the experimentation phase. Aditionally, some recommendations regarding the implementation of the client-server architecture are presented to be considered as future work.

\clearpage
