\chapter{Background}
\label{background}

Information security is a broad topic that covers many different aspects, from which two of them are particularly important for this work: confidentiality and authentication. There are some aspects from cryptography that would be relevant to review, such as asymmetric and symmetric cryptography, as well as the concepts: plaintext, ciphertext and key. A brief description of cloud computing and its relevant aspects are to be described, and how secure computing has an impact on it. Finally, the concept of homomorphic encryption will be explained, noting the mathematical properties and its categories of somewhat homomorphic encryption and fully homomorphic encryption.

Seen from a general perspective, Whitman and Herbert describe security as ``the quality or state of being secure---to be free from danger'' \cite{PrinciplesInformationSecurity}. Danger would refer to a potential harming action that an adversary can do, whether it is intentionally or not. Computer security is defined by the NIST Computer Security Handbook as follows:

``The protection afforded to an automated information system in order to attain the applicable objectives of preserving the integrity, availability, and confidentiality of information system resources (includes hardware, software, firmware, information/data, and telecommunications)''.

This definition introduces two relevant concepts in the previous definition, namely: information and confidentiality. RFC 2828 defines information as ``facts and ideas, whichcan be represented (encoded) as various forms of data,'' and data as ``information in a specific physical representation, usually a sequence of symbols that have meaning: especially a representation of information that can be processed or produced by a computer.'' Although both terms have different connotations depending on what it represents, both words are used interchangeably in this work. Now that information has been described as a concept, confidentiality can be introduced. More specifically, data confidentiality refers to the assurance that private or confidential information is not made available or disclosed to unauthorized individuals \cite{CryptoStallings}. This term is closely related to \textit{privacy}, which gives individuals the control of what information related to them may be accessed or stored by whom. Loss of confidentiality would imply unathorized access or disclosure of information.

Cryptography is the study of protecting data and communications. It involves communicating messages or information between two or more parties by changing the appearance of the messages, so that it becomes very difficult for unauthorized parties to intercept or interfere with the transmission of the information \cite{IntroCryptoMath}. Cryptography should not be confused with cryptology, because even though they overlap with each other, cryptology includes cryptanalysis as well. 

\cite{CryptoIT} approaches the study of cryptography by explaning the classical problem of transmitting secret messages from a sender A to a receiver B. Both the sender and receivers, A and B, can be thought of persons, organizations, or various technical systems.  They are formally described as abstract ``parties'' or ``entities'', but it is often more convenient to identify both of them as human participants that go by the names of Alice and Bob, instead of using the letters A and B.
For the messages to go from Alice to Bob, there has to be some kind of medium or channel where the information is transmitted. Generally, it is assumed that the channel can be potentially accessed by a third party that is neither the sender nor the receiver. On top of it, it is also assumed that either the receiver or sender has an adversary or enemy E, who is constantly trying to tamper or know the content of the messages passed between Alice and Bob. This attacker E who is constantly eavesdropping on the communication lines is usually called Eve, who is thought to have powerful computing facilities and is able to make use methods to learn of the message contents.  It is clear that both Alice and Bob want to protect their messages so that their contents are unclear to Eve. This implies a confidentiality goal, and it can be reached by using encryption algorithms, or as they are commonly known: ciphers.
Before a message is sent from Alice to Bob or viceversa, the sender \textit{encrypts} the message. In other words, a certain algorithm is applied on the message so that the content of it is unclear to a third party, such as Eve. Once the sent message has been received by the other party, he \textit{decrypts} it to recover the original content of the message before encryption, known as \textit{plaintext}. In order for this to work, Alice and Bob must have agreed in advance about various of the details needed, such as the algorithm and parameters used by the sender, so that the receiver knows how to perform the decryption properly. Since it would be troublesome that Eve learned of those details, Alice and Bob would have to use some sort of secure channel, such as trusted messngers or couriers, or set up a private meeting where no one else can eavesdrop. Making use of this secure channel, however, is more costly than using the regular communication channel between Alice and Bob, and it might not be available at all times, which is why it is recommended to use it only to agree on some details in advance, like the parameters for the encryption algorithm, as well as a \textit{private key}.




\clearpage
