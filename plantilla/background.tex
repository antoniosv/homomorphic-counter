\chapter{Background}
\label{background}

\textbf{Information Security and Data Confidentiality}

Information security is a broad topic that covers many different aspects, from which two of them are particularly important for this work: confidentiality and authentication. There are some aspects from cryptography that would be relevant to review, such as asymmetric and symmetric cryptography, as well as the concepts: plaintext, ciphertext and key. A brief description of cloud computing and its relevant aspects are to be described, and how can secure computations have an impact on it. Finally, the concept of homomorphic encryption will be explained, noting the mathematical properties and its categories of somewhat homomorphic encryption and fully homomorphic encryption.

Seen from a general perspective, Whitman and Herbert describe security as ``the quality or state of being secure---to be free from danger'' \cite{PrinciplesInformationSecurity}. Danger would refer to a potential harming action that an adversary can do, whether it is intentionally or not. Computer security is defined by the NIST Computer Security Handbook as follows:

``The protection afforded to an automated information system in order to attain the applicable objectives of preserving the integrity, availability, and confidentiality of information system resources (includes hardware, software, firmware, information/data, and telecommunications)''.

This definition introduces two relevant concepts in the previous definition, namely: information and confidentiality. RFC 2828 defines information as ``facts and ideas, whichcan be represented (encoded) as various forms of data,'' and data as ``information in a specific physical representation, usually a sequence of symbols that have meaning: especially a representation of information that can be processed or produced by a computer.'' Although both terms have different connotations depending on what it represents, both words are used interchangeably in this work. Now that information has been described as a concept, confidentiality can be introduced. More specifically, data confidentiality refers to the assurance that private or confidential information is not made available or disclosed to unauthorized individuals \cite{CryptoStallings}. This term is closely related to \textit{privacy}, which gives individuals the control of what information related to them may be accessed or stored by whom. Loss of confidentiality would imply unathorized access or disclosure of information.

\textbf{Cryptography Concepts}

Cryptography is the study of protecting data and communications. It involves communicating messages or information between two or more parties by changing the appearance of the messages, so that it becomes very difficult for unauthorized parties to intercept or interfere with the transmission of the information \cite{IntroCryptoMath}. Cryptography should not be confused with cryptology, because even though they overlap with each other, cryptology includes cryptanalysis as well. 

Other important aspects that have been linked to cryptography apart from confidentiality are \textit{authentication} and \textit{integrity}. Authentication gives the receiving parties the means to make sure that the source of the communication is really who it is thought to be, and not someone pretending to be the legitimate sender. On the other hand, integrity, among other things, ensures that the message received has not been tampered in any way. Different kind of cryptographic algorithms are required to satisfy these goals, and the same goes for confidentiality.

\cite{CryptoIT} approaches the study of cryptography by explaning the classical problem of transmitting secret messages from a sender A to a receiver B. Both the sender and receivers, A and B, can be thought of persons, organizations, or various technical systems.  They are formally described as abstract ``parties'' or ``entities'', but it is often more convenient to identify both of them as human participants that go by the names of Alice and Bob, instead of using the letters A and B.
For the messages to go from Alice to Bob, there has to be some kind of medium or channel where the information is transmitted. Generally, it is assumed that the channel can be potentially accessed by a third party that is neither the sender nor the receiver. On top of it, it is also assumed that either the receiver or sender has an adversary or enemy E, who is constantly trying to tamper or know the content of the messages passed between Alice and Bob. This attacker E who is constantly eavesdropping on the communication lines is usually called Eve, who is thought to have powerful computing facilities and is able to make use methods to learn of the message contents.  It is clear that both Alice and Bob want to protect their messages so that their contents are unclear to Eve. This implies a confidentiality goal, and it can be reached by using encryption algorithms, or as they are commonly known: ciphers.
Before a message is sent from Alice to Bob or viceversa, the sender \textit{encrypts} the message. In other words, a certain algorithm is applied on the message so that the content of it is unclear to a third party, such as Eve. Once the sent message has been received by the other party, he \textit{decrypts} it to recover the original content of the message before encryption, known as \textit{plaintext}. In order for this to work, Alice and Bob must have agreed in advance about various of the details needed, such as the algorithm and parameters used by the sender, so that the receiver knows how to perform the decryption properly. Since it would be troublesome that Eve learned of those details, Alice and Bob would have to use some sort of secure channel, such as trusted messngers or couriers, or set up a private meeting where no one else can eavesdrop. Making use of this secure channel, however, is more costly than using the regular communication channel between Alice and Bob, and it might not be available at all times, which is why it is recommended to use it only to agree on some details in advance, like the parameters for the encryption algorithm, as well as a \textit{private key}.
As Stanoyevitch \cite{IntroCryptoMath} describes it on his Introduction to Cryptography book, the above scheme would make up a cryptosystem, since it makes of cryptographic algorithms. As it has been hinted, such a cryptosystem has two parts: encryption, which is done by the sender as means of putting or transforming the actual \textbf{plaintext} into \textbf{ciphertext}, and \textbf{decryption}, which is done once the message has been received at the other end and has the purpose of translating the ciphertext back into the original plaintext message. Both encryption and decryption are usually done using a \textbf{key}, and it is intended that only the sender and receiver learn of the nature and value of the key. 

[INSERT BOB-ALICE-EVE DRAWING]

The use of the plaintext, ciphertext, and key concepts are recurrent throughout this work, and given their relevance, they are explained as follows, along with other concepts, the same way Cameron \cite{CryptoNotes} and Stallings \cite{CryptoStallings} describe them.
\begin{description}
\item[Plaintext] It is the original message that Alice wants to send to Bob. Neither Bob nor Alice want the message to be known to Eve. This piece of data is fed into the encryption algorithm before it is sent to the other end.
\item[Encryption algorithm] This algorithm uses a key that performs several substitutions and transformations on the plaintext. The result is a ciphertext that cannot be translated back to its original plaintext form without the appropriate key.
\item[Secret key] It is a value that both Alice and Bob agree upon, it is usually kept secret, since it is used by the encryption algorithm to obtain the ciphertext. The encryption algorithm will produce a distinct output depending on the key being used. In other words, if the secret key is leaked to Eve, this would mean that Eve can now learn any messages that are being produced using that key.
\item[Ciphertext] This is the modified or scrambled message produced as output by the encryption algorithm. Its value depends on the algorithm and plaintext, as well as the chosen key.
\item[Decryption algorithm] It is like the encryption algorithm, except that its run in reverse. The has two inputs: the ciphertext and its corresponding key. It then performs the necessary substitutions and transpositions so that the ciphertext is translated back into the original plaintext.
\end{description}

It is assumed at all times that Eve has access to the communication lines, and can always know the ciphertext that is being sent between Alice and Bob. It is also a good practice to assume that the encryption and decryption algorithms are known to Eve, which leaves the key as the most important part of the scheme. As mentioned previously, Alice and Bob need to establish the means to to agree on a private key, without anyone else knowing what this key is. As soon as the key is known to the attacker, i.e. Eve, all further communication is compromised. When that happens, another key has to be chosen and agreed upon once more by Alice and Bob.

The nature of the encryption and decryption can be classified in two broad categories: symmetric and asymmetric key encryption. \cite{CryptoKeys} describes both categories by looking at the keys used. In \textbf{symmetric encryption}, the key used for encryption is virtually the same used in decryption. Therefore, key distribution has to be done before the tranmission of the messages. The key length has a direct impact on the security offered by the encryption algorithm. Meanwhile, \textbf{asymmetric encryption} employs two different keys: public and private. The public key is closely tied to the receiver, and it is used for encryption of the data. It is called public because it is available for general use; therefore, Alice can use Bob's public key to send him messages. On the other hand, the private is kept secret from the outside world, and is only available by an authorized person. The public and private keys always come in pairs, and this is because while the public key is used to encrypt data; its counterpart, the private key, is used to decrypt the data. 

\begin{comment}
\textbf{Security in the Cloud}

The cloud, a trendy term used to describe a network of servers usually accessible through the Internet, currently provides two important services: storage and computation. 

\textit{Cloud storage}

\textit{Proof of storage}

\textit{Cloud Computing}

Li, Jin et. al \cite{cryptoeprint:2009:593} introduce cloud computing as a new term that views computing as an utility which enables convenient, on-demand access to computing resources that can be rapidly deployed with minimal management overhead and great efficiency. Cloud computing has the potential to benefit its users in avoiding large capital expenditure used for deployment and management of both software and hardware, such as setting up a data center. 

\textit{Key management}}


\textbf{Secure computation}

\textit{Multiparty computation}

\textit{Introduction to homomorphic cryptography}


\textbf{Homomorphic Encryption}

\textit{Intro}

\textit{Classification}

\textit{Mathematical things}

\textit{Intro to lattice-based crypto}

\textit{BGV scheme}

\end{comment}

\clearpage
