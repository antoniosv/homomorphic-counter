\chapter{Abstract}
\markboth{Resumen}{}

\noindent\autor.

\noindent Candidato para el grado de \grado.
%\indent 

\noindent\uanl.\\
\noindent\fime.

\noindent T\'{\i}tulo del estudio: 

\begin{center}
\begin{tabular}{p{11cm}}
	\centering
	\scshape{\large{\titulo}}
\end{tabular}
\end{center}\bigskip

\noindent N\'{u}mero de p\'{a}ginas: \pageref{LastPage}.

\paragraph{Objetives and Study Method}
%Problem
This work proposes an implementation of a client-server architecture based software that uses HElib to enable homomorphic encryption and perform computations on the encrypted data. Cloud computing is a cost-efficient alternative to host data remotely and subsequently perform computations on it. Confidentiality becomes an issue when data is delegated to the cloud, as often the information hosted is considered to be sensitive. A typical approach is to make use of a cryptographic algorithm that encrypts the data with a secret key that only the owner has. However, as long as the owner of the data wishes to make changes to the encrypted data, he would have to download the encrypted data, decrypt it using his secret key, perform the desired computations or changes to it, and reencrypt it once again before uploading it to the server in the cloud. An alternative to this approach is to make use of \emph{homomorphic encryption}, an advanced technique in cryptography that enables computations on encrypted data without needing the secret key. 

%Motivation
The motivation behind using homomorphic encryption instead of employing the typical approach is that it would not require that the encrypted data is decrypted prior to its modification, as well as it would not have to be reencrypted before uploading to the cloud, saving processing time and bandwidth. However, so far the schemes that make this possible have been thought of as being too slow and expensive, storage wise. A library that implements homomorphic encryption, HElib, is based off an homomorphic encryption scheme that seems very promising because of the optimizations and improvements over other past schemes. Using HELib, it would be possible to reevaluate the situation and show to what degree it would be feasible to make a solution based on homomorphic encryption. Therefore, it is worthwhile to make an effort to put this library into practice and measure how its use fares in a real application.

%Lo que proponemos
The proposed solution consists in taking the functionalities provided by HElib, and build a client-server based software that aims to address the main scenarios found in the proposed case study. Even though the proposed implementation could in theory work for a number of examples, it is designed taking into account a case study; therefore, both are closely related.

%Caso de Estudio
Regarding our case study, consider the scenario where a household has an expected pattern of activity, i.e. empty during the day and non-empty at night, so that the householder seeks to ascertain the number of people inside at any point in time during the day. Assuming the householder has put in place certain sensors around the building so that it detects and counts who comes in and out, he would like to learn the value of the counter remotely. As the householder chooses to store the value in the cloud, he quickly realizes he does not want others to learn of this value, not even the cloud service itself, as to prevent potential burglars to break in when the household is empty. Additionally, he wishes that the counter gets updated at certain intervals, e.g. every 20 minutes. 

%Problematica al encriptar cada vez
As it was briefly mentioned, a problem found in the usual approach of re-encrypting and reuploading the data every time a change is made, is that the overhead costs might be too high to do this often. For one, bandwidth would be spent on each ocassion, and the time required for encryption and uploading might not be that short. Therefore, for this particular case study, an approach that relies on homomorphic encryption is chosen.

\paragraph{Contributions and Conclusions}
%Diferencia de los demas trabajos
In contrast with other works, the proposed solution tackles the problem by employing homomorphic encryption. It is a simple client-server architecture which could be expanded to adapt with other scenarios and requirements. The proof of concept provides the means to operate on the encrypted data, i.e. performing additions on the counter value, without compromising confidentiality and wasting resources caused by the overhead costs of re-encrypting data at every change.

%Experimentos

To show the feasibility of using HElib to do homomorphic encryption, experiments were designed by varying the value of a security parameter $k$. This value has a direct impact on the execution time of key generation, encryption, and decryption, as well as the size of the resulting public key. The findings show that even though key generation time is long, taking approximately between 9 and 13 seconds, and key size is large, being between 35 megabytes and 165 megabytes, it is not considered to be an issue, as this step is done one time only. On the other hand, encryption, decryption, and addition times are amazingly fast, none of them exceding over 1 second. However, the ciphertext raises some doubts over the magnitude of its size: during development, it was found that on average, ciphertext size was 74 megabytes; meanwhile, during experimentation, it was found that ciphertext size rose no higher than 45kB. To the best of our knowledge, there is yet no way to predict the size of the ciphertext using HElib. Therefore, if it was decided to keep the size of the ciphertext below 45kB, it would certainly be feasible to take this solution as an application on the cloud; otherwise, further evaluation on the impact would be required.

\noindent Signature of Advisor \rule{72mm}{0.3pt}

\vspace*{-4mm}

\noindent \phantom{Advisor signature: m} \asesor
